\section*{Тема работы}

Программно-алгоритмическая реализация метода Рунге-Кутта 4-го порядка точности при решении системы ОДУ в задаче Коши.

\section*{Цель работы}
Получение навыков разработки алгоритмов решения задачи Коши при реализации моделей, построенных на системе ОДУ, с использованием метода Рунге-Кутта 4-го порядка точности.

\chapter{Теоретические сведения}

Опишем колебательный контур с помощью системы уравнений:

\begin{equation*}
    \begin{cases} L_k \frac{dI}{dt} + (R_k + R_p(I)) \cdot I - U_C = 0
        \\ \frac{dU_c}{dt} = - \frac{I}{C_k}
    \end{cases}
\end{equation*}\\

Значение $R_p(I)$ можно вычислить по формуле:\\

$R_p = \frac{l_e}{2 \pi \cdot \int_0^R \sigma(T(r))rdr} = \frac{l_e}{2 \pi R^2 \cdot \int_0^1 \sigma(T(z))dz}$\\

т. к. $z=r/R$. \\

\indent Значение $T(z)$ вычисляется по формуле:\\

$T(z) = T_0 + (T_w - T_0) \cdot Z^m$\\

Заданы начальные параметры:\\\\
\indent $R$ = 0.35 см (Радиус трубки)\\
\indent $l_{e}$ = 12 см (Расстояние между электродами лампы)\\
\indent $L_{k}$ = 187e-6 Гн (Индуктивность)\\
\indent $C_{k}$ = 268e-6 Ф (Емкость конденсатора)\\
\indent $R_{k}$ = 0.25 Ом (Сопротивление)\\
\indent $U_{c0}$ = 1400 В (Напряжение на конденсаторе в начальный момент времени)\\
\indent $I_{0}$ = 0..3 А (Сила тока в цепи в начальный момент времени t = 0)\\
\indent $T_{w}$ = 2000 K

\section{Метод Рунге-Кутта четвертого порядка точности}

Имеем систему уравнений вида:
\begin{equation*}
    \begin{cases}
        u'(x) = f(x, u(x)) \\
        u(\xi) = \eta
    \end{cases}
\end{equation*}

Тогда:\newline

$y_{n+1} = y_n + \frac{k_1 + 2k_2 + 2k_3 + k_4}{6}$

$k_1 = h_n f(x_n, y_n)$

$k_2 = h_n f(x_n + \frac{h_n}{2}, y_n + \frac{k_1}{2})$

$k_3 = h_n f(x_n + \frac{h_n}{2}, y_n + \frac{k_2}{2})$

$k_4 = h_n f(x_n + h_n, y_n + k_3)$\newline

Рассмотрим обобщение формулы на случай двух переменных. Пусть дана система:

\begin{equation*}
    \begin{cases}
        u'(x) = f(x, u, v) \\
        v'(x) = \varphi(x, u, v) \\
        v(\xi) = v_0 \\
        u(\xi) = u_0 \\
    \end{cases}
\end{equation*}

Тогда:\newline

$y_{n+1} = y_n + \frac{k_1 + 2k_2 + 2k_3 + k_4}{6}$

$z_{n+1} = z_n + \frac{q_1 + 2q_2 + 2q_3 + q_4}{6}$

$k_1 = h_n f(x_n, y_n, z_n)$

$k_2 = h_n f(x_n + \frac{h_n}{2}, y_n + \frac{k_1}{2}, z_n + \frac{q_1}{2})$

$k_3 = h_n f(x_n + \frac{h_n}{2}, y_n + \frac{k_2}{2}, z_n + \frac{q_2}{2})$

$k_4 = h_n f(x_n + h_n, y_n + k_3, z_n + q_3)$

$q_1 = h_n \varphi(x_n, y_n, z_n)$

$q_2 = h_n \varphi(x_n + \frac{h_n}{2}, y_n + \frac{k_1}{2}, z_n + \frac{q_1}{2})$

$q_3 = h_n \varphi(x_n + \frac{h_n}{2}, y_n + \frac{k_2}{2}, z_n + \frac{q_2}{2})$

$q_4 = h_n \varphi(x_n + h_n, y_n + k_3, z_n + q_3)$

\chapter{Реализация}

\section{Код программы}

Ниже представлены исходные коды программы на языке \texttt{Elixir}.

\begin{lstinputlisting}[
        caption={Основной модуль приложения},
        style={mystyle},
    ]{../lib/application.ex}
\end{lstinputlisting}

\begin{lstinputlisting}[
        caption={Модуль с интерполяционными функциями},
        style={mystyle},
    ]{../lib/interpolated_funcs.ex}
\end{lstinputlisting}

\begin{lstinputlisting}[
        caption={Модуль с функциями прогонки с разными параметрами},
        style={mystyle},
    ]{../lib/runners.ex}
\end{lstinputlisting}

\begin{lstinputlisting}[
        caption={Модуль с методами численного интегрирования},
        style={mystyle},
    ]{../lib/integral.ex}
\end{lstinputlisting}

\begin{lstinputlisting}[
        caption={Модуль с генераций искомых значений},
        style={mystyle},
    ]{../lib/solver.ex}
\end{lstinputlisting}

\begin{lstinputlisting}[
        caption={Модуль с функциями тока и напряжения},
        style={mystyle},
    ]{../lib/main_funcs.ex}
\end{lstinputlisting}

\section{Результаты работы программы}

На рисунках \ref{img:1_1} -- \ref{img:1_5} представлены графики зависимости от времени импульса $t$: $I(t)$, $U(t)$, $R_{p}(t)$, $I(t) \cdot R_{p}(t)$, $T_{0}(t)$ при исходных данных. Интервал: $[0, 0.0008]$, шаг $h = 10^{-6}$.

\imgw{140mm}{1_1}{График $I(t)$}
\clearpage
\imgw{120mm}{1_2}{График $U(t)$}
\imgw{120mm}{1_3}{График $R_p(t)$}
\clearpage
\imgw{120mm}{1_4}{График $T0(t)$}
\imgw{120mm}{1_5}{График $I(t) \cdot R_p(t)$}
\clearpage

На рисунке \ref{img:2} представлен график $I(t)$, при $R_{k} + R_{p} = 0$. Интервал: [0, 0.0008], шаг $h = 10^{-6}$.

\imgw{100mm}{2}{График зависимости $I(t)$ при $R = 0$}

На рисунке \ref{img:3} представлен график $I(t)$, при $R_{k} + R_{p} = 200$. Интервал: [0, 0.00002], шаг $h = 10^{-7}$.

\imgw{100mm}{3}{График зависимости $I(t)$ при $R = 200$}

На рисунках \ref{img:4_1} - \ref{img:4_7} представлены результаты исследования влияния параметров контура $C_{k}, L_{k}$ и $R_{k}$ на длительность импульса $t$.

\clearpage
\imgw{110mm}{4_1}{График зависимости $I(t)$ при начальных параметрах}
\imgw{110mm}{4_2}{График зависимости $I(t)$ при увеличении начального значения $C_{k}$ в 2 раза.}
\clearpage
\imgw{110mm}{4_3}{График зависимости $I(t)$ при уменьшении начального значения $C_{k}$ в 1.5 раза.}
\imgw{110mm}{4_4}{График зависимости $I(t)$ при увеличении начального значения $L_{k}$ в 2 раза}
\clearpage
\imgw{110mm}{4_5}{График зависимости $I(t)$ при уменьшении начального значения $L_{k}$ в 2 раза}
\imgw{110mm}{4_6}{График зависимости $I(t)$ при увеличении начального значения $R_{k}$ в 10 раз}
\clearpage
\imgw{110mm}{4_7}{График зависимости $I(t)$ при уменьшении начального значения $R_{k}$ в 10 раз}

Выводы:
\begin{itemize}
    \item увеличение $C_{k}$ приводит к увеличению длительности импульса $t$;
    \item уменьшение $C_{k}$ приводит к уменьшению длительности импульса $t$;
    \item увеличение $L_{k}$ приводит к увеличению длительности импульса $t$;
    \item уменьшение $L_{k}$ приводит к уменьшению длительности импульса $t$;
    \item увеличение $R_{k}$ приводит к увеличению длительности импульса $t$;
    \item уменьшение $R_{k}$ приводит к уменьшению длительности импульса $t$;
\end{itemize}

\chapter{Ответы на вопросы}

\subsection*{1. Какие способы тестирования программы, кроме указанного в п. 2, можете предложить ещё?}\\

В цепи мы можем влиять на 3 параметра (ёмкость, индуктивность, сопротивление), при чем влияние каждого параметра по отдельности мы рассмотрели в п.2. При этом особый интерес представляет изменение сопротивления (потому что изменение ёмкости и индуктивности по большей мере влияют на изменение длительности импульса (так, при нулевом сопротивлении, график всегда будет представлен синусоидой, а изменение параметров будет влиять только на значение ее экстремумов и период)) путем добавления/удаления/изменения объектов, обладающих сопротивлением (лампа, резистор), при этом: если $R_{k}$ мало, то возникнут затухающие колебания, если велико $R_{k}$ --- апериодическое затухание.\\

\subsection*{2. Получите систему разностных уравнений для решения сформулированной задачи неявным методом трапеций. Опишите  алгоритм реализации полученных уравнений.}

\begin{equation}
    U_{n + 1} = U_{n} + \frac{h}{2} + f(x_{n}, u_{n}) + f(x_{n + 1}, u_{n + 1}) + O(h^2)
\end{equation}

\begin{equation}
    \begin{cases}
        \frac{dI}{dT} = \frac{U - (R_{k} + R_{p}(I))I}{L_{k}}\\
        \frac{dU}{dt} = -\frac{I}{C_{k}}\\
    \end{cases}
\end{equation}

\begin{equation}
    I_{n + 1} = I_{n} + \frac{h}{2} (\frac{U_{n} - (R_{k} + R_{p}(I_{n}))I_{n}}{L_{k}} + \frac{U_{n + 1} - (R_{k} + R_{p}(I_{n + 1}))I_{n + 1}}{L_{k}})
\end{equation}

\begin{equation}
    U_{n + 1} = U_{n} + \frac{h}{2}(-\frac{I_{n}}{C_{k}} -\frac{I_{n + 1}}{C_{k}}) = U_{n} - \frac{h}{2}(\frac{I_{n} + I_{n + 1}}{C_{k}})
\end{equation}

Подставляя (\textbf{3.4}) в (\textbf{3.3}), имеем:

\begin{equation}
    I_{n + 1} = I_{n} + \frac{h}{2L_{k}}(2U_{n} -  (R_{k} + R_{p}(I_{n}) + \frac{h}{2C_{k}})I_{n} - (R_{k} + R_{p}(I_{n + 1}) + \frac{h}{2C_{k}})I_{n+1})
\end{equation}\\

\subsection*{3. Из каких соображений проводится выбор численного метода того или иного порядка точности, учитывая, что чем выше порядок точности метода, тем он более сложени требует, как правило, больших ресурсов вычислительной системы?}\\

Выбор численного метода проводится исходя из требований поставленной задачи, объема вычислений, а также от свойств функций, используемых в вычислениях.

Оценивается погрешность для частного случая вида правой части дифференциального уравнения: $\varphi (x, \nu) = \varphi(x)$

Так, если $\varphi (x, \nu)$ непрерывна и ограничена и ограничены и непрерывны её четвертые производные, то наилучший результат достигаем при использовании метода Рунге-Кутта 4 порядка:\\

$y_{n + 1} = y_{n} + \frac{k_{1} + 2k_{2} + 2k_{3} + k_{4}}{6}$, где\\

$k_{1} = h_{n} f(x_{n}, y_{n})$\\
\indent $k_{2} = h_{n}f(x_{n} + \frac{h_{n})}{2}, y_{n} + \frac{k_{1}}{2}$\\
\indent $k_{3} = h_{n}f(x_{n} + \frac{h_{n})}{2}, y_{n} + \frac{k_{2}}{2}$\\
\indent $k_{4} = h_{n}f(x_{n} + \frac{h_{n})}{2}, y_{n} + k_{3}$\\

Однако в случае если $\varphi (x, \nu)$ не имеет таких производных, то четвертый порядок схемы не может быть достигнут и стоит применять более простые схемы.
