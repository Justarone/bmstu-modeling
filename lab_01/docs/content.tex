\chapter{Теоретические сведения}

\section{Цель работы}
Получение навыков решения задачи Коши при помощи метода Пикара, метода Эйлера и метода Рунге-Кутты 2-го порядка.

\section{Задача}

Имеем ОДУ, у которого отсутствует аналитическое решение:

\begin{equation}
    \label{initial_odu}
    \begin{cases}
        u'(x) = f(x, u)\\
        u(\xi) = \eta
    \end{cases}
\end{equation}\newline

Для решения данного ОДУ были использованы 3 алгоритма.

\subsection{Метод Пикара}

Имеем:

\begin{equation}
    \label{solution}
    u(x) = \eta +  \int_{\xi}^{x} f(t,u(t)) \,dt
\end{equation}

Строим ряд функций:

\begin{equation}
    \label{sol}
    y^{(s)} = \eta +  \int_{\xi}^{x} f(t,y^{(s-1)}(t)) \,dt, \quad \quad
    y^{(0)} = \eta
\end{equation}

Построим 4 приближения для уравнения (\ref{solution}):

\begin{equation}
    \label{f1}
    y^{(1)}(x) = 0 + \int_{0}^{x} t^2 \,dt = \frac{x^3}{3}
\end{equation}

\begin{equation}
    \label{f2}
    y^{(2)}(x) = 0 + \int_{0}^{x} (t^2 + \left(\frac{t^3}{3}\right)^2) \,dt = \frac{x^3}{3} + \frac{x^7}{63}
\end{equation}

\begin{equation}
    \label{f3}
    y^{(3)}(x) = 0 + \int_{0}^{x} (t^2 + \left(\frac{t^3}{3} + \frac{t^7}{63}\right)^2) \,dt = \frac{x^3}{3} + \frac{x^7}{63} + \frac{2x^{11}}{2079} + \frac{x^{15}}{59535}
\end{equation}

\begin{equation}
    \begin{split}
        \label{f4}
        y^{(4)}(x) = 0 + \int_{0}^{x} (t^2 + \left(\frac{t^3}{3} + \frac{t^7}{63} + \frac{2t^{11}}{2079} + \frac{t^{15}}{59535}\right)^2) \,dt = \frac{x^3}{3} + \frac{x^7}{63} + \frac{2x^{11}}{2079} +\\
        \frac{x^{15}}{59535} + \frac{2x^{15}}{93555} + \frac{2x^{19}}{3393495} + \frac{2x^{19}}{2488563} + \frac{2x^{23}}{86266215} + \\
        \frac{x^{23}}{99411543} + \frac{2x^{27}}{3341878155}  + \frac{x^{31}}{109876902975}
    \end{split}
\end{equation}

\subsection{Метод Эйлера}

\begin{equation}
    \label{ey}
    y^{(n+1)}(x) = y^{(n)}(x) + h \cdot f(x_{n}, y^{(n)})
\end{equation}

\indent Порядок точности: $O(h)$.

\subsection{Метод Рунге-Кутта}

\begin{equation}
    \label{rk}
    y^{n+1}(x) = y^{n}(x) + h ((1-\alpha) R_1 + \alpha R_2)
\end{equation}\newline

где $R1 = f(x_{n}, y^{n})$, $R2 = f(x_{n} + \frac{h}{2\alpha}, y^{n} + \frac{h}{2\alpha}R_1)$, $\alpha = \frac{1}{2}$ или 1\newline

Порядок точности: $O(h^2)$.

\clearpage

\chapter{Реализация}

Ниже представлены исходные коды программы на языке Elixir.

\begin{lstinputlisting}[
        caption={Основной модуль приложения},
        style={mystyle},
    ]{../lib/application.ex}
\end{lstinputlisting}

\begin{lstinputlisting}[
        caption={Функции для работы с многочленами},
        style={mystyle},
    ]{../lib/polynomial.ex}
\end{lstinputlisting}

\begin{lstinputlisting}[
        caption={Модуль реализации метода Пикара},
        style={mystyle},
    ]{../lib/solvers/main.ex}
\end{lstinputlisting}

\begin{lstinputlisting}[
        caption={Модуль реализации методов Эйлера и Рунге-Кутта},
        style={mystyle},
    ]{../lib/solvers/other.ex}
\end{lstinputlisting}

\begin{lstinputlisting}[
        caption={Вспомогательный модуль чтения ввода},
        style={mystyle},
    ]{../lib/parser.ex}
\end{lstinputlisting}

\begin{lstinputlisting}[
        caption={Модуль со вспомогательными функциями},
        style={mystyle},
    ]{../lib/helper.ex}
\end{lstinputlisting}

\chapter{Результат работы программы}

Ниже приведена таблица с вычисленными значениями. Значения вычислены для шага $10^{-4}$, выведено каждое 500 значение.

\imgw{170mm}{table}{Таблица с вычисленными значениями}

\chapter{Ответы на контрольные вопросы}

\section{Укажите интервалы значений аргумента, в которых можно считать решением заданного уравнения каждое из первых 4-х приближений Пикара. Точность результата оценивать до второй цифры после запятой. Объяснить свой ответ.}

Ответ на данный вопрос даётся с опорой на значения из таблицы выше (Шаг равен $10^{-4}$). Учитывая тот факт, что $i$-ое приближение имеет порядок точности $= O(h^{i})$, можно сделать вывод, что метод Пикара $i$-ого порядка точности будет точен до тех пор, пока его значение совпадает (с определенной степенью точности, в нашем случае - 2 цифры после запятой) с методом Пикара $(i+1)$-ого порядка точности. Как только это значение начинает отклоняться, учитывая тот факт, что более высокое приближение имеет меньшую погрешность, метод нельзя считать решением.

Таким образом:
\begin{enumerate}
    \item 1 приближение считается решением на полуинтервале $[0; 0.85)$, потому что в точке 0.85 его значение отличиается во 2 цифре после запятой;
    \item 2 приближение считается решением на полуинтервале $[0; 1.20)$;
    \item 3 приближение считается решением на полуинтервале $[0; 1.40)$;
    \item 4 приближение считается решением на отрезке $[0; 1.40]$, при этом дать более точную оценку для этого интервала без наличия 5 приближения невозможно;
\end{enumerate}

\section{Пояснить, каким образом можно доказать правильность полученного результата при фиксированном значении аргумента в численных методах.}

Чтобы доказать правильность полученного численными методами результата, требуется устремить шаг к 0 до тех пор, пока не получится ситуации, когда при изменении шага значение функции не будет меняться (или будет меняться незначительно, в пределах допустимых в задачи пределов). При достижении такой ситуации, полученное значение можно считать правильным.

\section{Каково значение функции при $x=2$, т.е. привести значение $u(2)$.}

Ответ: $\approx 317$.
